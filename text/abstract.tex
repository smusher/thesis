ATP-sensitive potassium (K\ATP{}) channels are present in many tissues, most notably pancreatic islets and cardiac cells, where they couple the metabolic state of a cell to its electrical activity by regulating the flow of K\textsuperscript{} across the membrane in response to the intracellular ATP/ADP ratio.
K\ATP{} channels are an octameric complex, comprised of four inwardly-rectifying potassium channel (Kir) subunits, each of which is associated with a sulphonylurea receptor (SUR) subunit. 
In pancreatic islets, K\ATP{} channels are formed by Kir6.2 and SUR1.

The physiological regulation of K\ATP{} activity by the ATP/ADP ratio is the summed contribution of activation by Mg-nucleotides binding to SUR1, and inhibition by nucleotides binding to Kir6.2.
Mutations in either Kir6.2 and SUR1 which lead to diseases of insulin secretion are frequently observed to disrupt the nucleotide regulation of the channel.
